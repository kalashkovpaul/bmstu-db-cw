\section*{\large ЗАКЛЮЧЕНИЕ}
\addcontentsline{toc}{section}{ЗАКЛЮЧЕНИЕ}

В  ходе  выполнения  курсовой работы  было  разработано  программное обеспечение, предназначенное для использования как информационная система для частных поликлиник.

Целью работы достигнута, были выполнены следующие задачи:
\begin{enumerate}[label=\arabic*)]
	\item проведён анализ предметной области;
	\item определён функционал, реализуемый информационной системой;
%	\item изучить существующие на рынке решения;
	\item спроектирована и разработана базу данных в соответствии с поставленной задачей;
	\item спроектировано и разработано приложение для работы с созданной базой данных;
	\item исследована зависимость времени обработки запроса от типа подключения к базе данных.
\end{enumerate}

Разработанная программа позволяет просматривать методы созданного API, их подробные характеристики, а также осуществлять запросы после ввода необходимых аргументов.
Проведён анализ времени обработки запросов с использованием чистого SQL, ORM и Query Builder, из результатов следует, что использование чистого SQL является самым быстрым способом (1.78 мс), однако не намного превосходит время, затраченное на выполнение SQL при использовании ORM (на 0.18 мс). Использование Query Builder занимает больше всего времени (12.79 мс).


\pagebreak