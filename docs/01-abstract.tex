\section*{\large РЕФЕРАТ}

TODO что написать в задании?
Расчетно-пояснительная записка \pageref{LastPage} с., \totalfigures\ рис., \totaltables\ табл., 22 ист.

БАЗЫ ДАННЫХ, PostgreSQL, РЕЛЯЦИОННАЯ МОДЕЛЬ, REST


Цель работы --- разработка информационной системы для частных поликлиник.

Моделью базы данных является реляционная модель, архитектура приложения --- двухзвенная клиент-серверная.
Для представления API используется паттерн REST.

Результаты: разработана программа, предназначенная для использования сотрудниками частной поликлиники. Предусмотрены различные роли, соответствующие должностям минимального штата поликлиники, возможность администрирования приложения и работе с информацией о пациентах и приёмах.

<<<<<<< Updated upstream
Проведён анализ времени обработки запросов с использованием чистого SQL, ORM и Query Builder, из результатов следует, что использование чистого SQL является самым быстрым способом (1.78 мс), однако не намного превосходит время, затраченное на выполнение SQL при использовании ORM (на 0.18 мс). Использование Query Builder занимает больше всего времени (12.79 мс).
=======
TODO:

Проведено исследование зависимости времени обработки запроса от количества полей таблицы.
Из результатов следует, что зависимость времени обработки запроса от количества полей таблицы имеет линейный характер.
>>>>>>> Stashed changes

\pagebreak