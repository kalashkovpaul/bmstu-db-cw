\section*{ВВЕДЕНИЕ}
\addcontentsline{toc}{section}{ВВЕДЕНИЕ}

В XXI веке медицинское обслуживание является неотъемлемой частью жизни каждого человека. Несмотря на то, что большая часть населения России пользуется муниципальными учреждениями здравоохранения, лишь половина обращавшихся остаётся удоволетворена качеством и скоростью полученной медицинской помощи (по данным Всероссийского центра изучения общественного мнения)~\cite{wciom}.

Треть россиян посещает частные поликлиники~\cite{wciom}, что приводит к накоплению огромных объёмов данных, связанных с медицинским обслуживанием пациентов. 
Эффективность использования работниками поликлиник влияет на качество оказываемой медицинской помощи. 
Таким образом, создание информационной системы для частныхи поликлиник является актуальной задачей на 2023 год.


\textbf{Целью работы}: разработка информационной системы для частных поликлиник.
Для достижения поставленной цели необходимо выполнить следующие задачи:
\begin{enumerate}[label=\arabic*)]
	\item провести анализ предметной области;
	\item определить функционал, реализуемый информационной системой;
	\item изучить существующие на рынке решения;
	\item спроектировать и разработать информационную систему в соответствии с поставленной задачей;
	\item исследовать зависимость RPS от количества потоков нагрузочного тестирования.
\end{enumerate}

