\section{Технологическая часть}

В  данном  разделе  рассмотрены  средства  разработки  программного
обеспечения, приведены детали реализации и листинги исходных кодов
программы, а также приведён пример результата работы программы.

\subsection{Средства реализации}
Как  основное  средство  реализации  и  разработки  ПО  был  выбран  язык
программирования  TypeScript \cite{ts}.
Причиной  выбора  данного  языка  является  тот факт,  что  он  имеет компилируется в JavaScript \cite{js}, что делает его также кроссплатформенным и позволяет использовать библиотеки, написанные для JavaScript. 
Наличие типизации позволяет траспилятору получать более эффективный код, чем для JavaScript, вследствие чего производительность TypeScript является предпочтительнее при написании сервера \cite{ts-vs-js}.

Для запуска написанного сервера была использована библиотека Node.js \cite{nodejs} для языка JavaScript, которая предоставляет асинхронное окружение, что позволяет избавиться от синхронности и разрабатывать более производительные приложения.

Для обработки HTTP запросов была использована библиотека Fastify \cite{fastify}, которая была выбрана из-за своего превосходства в скорости обработки соединений по сравнению с другими библиотеками (Koa, Express, Restify, Hapi) \cite{fastify-benchmarks}.
Для доступа к СУБД, реализованной при помощи PostgreSQL \cite{postgres}, была использована библиотека pg-promise \cite{pg-promise}, предоставляющая интерфейс для взаимодействия с базой данных.

Для реализации графического итерфейса API была использована библиотека FastAPI \cite{fastapi}, позволяющая быстро создать рабочий интерфейс для созданного API (вместе с документацией).
Средой разработки  послужил  графический  редактор  Visual  Studio  Code \cite{vscode},  который известен  содержанием  большого  количество  плагинов,  ускоряющих  процесс разработки  программы.


\subsection{Реализация сервера}

В  расположенных  ниже  листингах \ref{code:1-1} -- \ref{code:2-3} приведены реализации менеджера соединений и класса соединения, объекты которого используются в обработчиках запросов.
\clearpage

\begin{code}
	\captionsetup{justification=centering}
	\captionof{listing}{Реализация менеджера соединений (часть 1)}
	\label{code:1-1}
	\inputminted
	[
	frame=single,
	framerule=0.5pt,
	framesep=20pt,
	fontsize=\small,
	tabsize=4,
	linenos,
	numbersep=5pt,
	xleftmargin=10pt,
	]
	{text}
	{code/ConnectManager1.ts}
\end{code}

\clearpage

\begin{code}
	\captionsetup{justification=centering}
	\captionof{listing}{Реализация менеджера соединений (часть 2)}
	\label{code:1-2}
	\inputminted
	[
	frame=single,
	framerule=0.5pt,
	framesep=20pt,
	fontsize=\small,
	tabsize=4,
	linenos,
	numbersep=5pt,
	xleftmargin=10pt,
	]
	{text}
	{code/ConnectManager2.ts}
\end{code}

\begin{code}
	\captionsetup{justification=centering}
	\captionof{listing}{Реализация класса соединения (часть 1)}
	\label{code:2-1}
	\inputminted
	[
	frame=single,
	framerule=0.5pt,
	framesep=20pt,
	fontsize=\small,
	tabsize=4,
	linenos,
	numbersep=5pt,
	xleftmargin=10pt,
	]
	{text}
	{code/Connection1.ts}
\end{code}
\clearpage

\begin{code}
	\captionsetup{justification=centering}
	\captionof{listing}{Реализация класса соединения (часть 2)}
	\label{code:2-2}
	\inputminted
	[
	frame=single,
	framerule=0.5pt,
	framesep=20pt,
	fontsize=\small,
	tabsize=4,
	linenos,
	numbersep=5pt,
	xleftmargin=10pt,
	]
	{text}
	{code/Connection2.ts}
\end{code}

\clearpage

\begin{code}
	\captionsetup{justification=centering}
	\captionof{listing}{Реализация класса соединения (часть 3)}
	\label{code:2-3}
	\inputminted
	[
	frame=single,
	framerule=0.5pt,
	framesep=20pt,
	fontsize=\small,
	tabsize=4,
	linenos,
	numbersep=5pt,
	xleftmargin=10pt,
	]
	{text}
	{code/Connection2.ts}
\end{code}


\subsection*{Вывод}
В данном разделе были рассмотрены средства реализации ПО, приведены
листинги исходного кода программы, разработанной на основе
алгоритмов,  выбранных  в  аналитическом  разделе  и  соответствующих приведённой в конструкторской части диаграмме классов приложения.

\pagebreak