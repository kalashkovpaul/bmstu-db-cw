\section{\large Исследовательская часть}
\label{cha:research}

В данном разделе будут приведены примеры работы разработанной программы и проведён анализ времени выполнения запросов с использованием: чистого SQL, ORM (Object-Relational Mapping) и Query Builder. Результаты иссследования позволят точнее определять основные требования к базе данных при её выборе для проектирования системы, тем самым облегчая процесс выбора наиболее подходящей для конкретной задачи СУБД.	

%\subsection{Результаты разработки}

\subsection{Описание интерфейса}

В основной части страницы находятся методы API, а также их особенности (тип метода, адрес методы, структура, которую метод использует). Часть таких методов приведена на рисунке \ref{fig:interface-1}). 
\begin{figure}[h]
	\centering
	\captionsetup{justification=centering}
	\includegraphics[width=170mm]{img/interface.png}
	\caption{Интерфейс программы: методы API}
	\label{fig:interface-1}
\end{figure}

По нажатию на элемент каждого метода появляется подробное описание: какие данные метод принимает на вход, какие параметры являются обязательными, а какие опциональными, структуры какого вида возвращает и в каком виде, что возвращает в случае ошибки (рисунок \ref{fig:interface-2}).
\begin{figure}[h]
	\centering
	\captionsetup{justification=centering}
	\includegraphics[width=170mm]{img/interface-2.png}
	\caption{Интерфейс программы: подробное описание метода API}
	\label{fig:interface-2}
\end{figure}


\subsection{Демонстрация работы программы}

На рисунках \ref{fig:example-1} -- \ref{fig:example-2} продемонстрирована работа программы на примере осуществления запроса для получения списка пациентов. Созданный интерфейс позволяет не только просматривать методы API, как было показано выше, но и осуществлять отправку конкретных запросов.
При этом имеется возможность посмотреть результат запроса (его статус и заголовки, присланные данные), его адрес, а также получить команды для отправления эквивалентного запроса из командной строки.

\clearpage

\begin{figure}[h]
	\centering
	\captionsetup{justification=centering}
	\includegraphics[width=152mm]{img/example-1.png}
	\caption{Демонстрация  работы  программы  (часть 1)}
	\label{fig:example-1}
\end{figure}

\begin{figure}[h]
	\centering
	\captionsetup{justification=centering}
 	\includegraphics[width=152mm]{img/example-2.png}
	\caption{Демонстрация  работы  программы  (часть 2)}
	\label{fig:example-2}
\end{figure}

\subsection{Постановка эксперимента}

Целью эксперимента является сравнение времени обработки запроса \linebreak (GET, на примере получения списка пациентов), выполненного при помощи чистого SQL, ORM и Query Builder. Для запросов, выполненных при помощи ORM и Query Builder необходимо дополнительно сравнить время, в течение которого выполняются SQL запросы. Необходимо учесть кэширование буфера (очищать кэш перед проведением замеров), а также учесть влияние прогрева буферного кэша на результаты замеров (не учитывать первый запрос, во время которого происходит прогрев кэша).

Технические характеристики устройства, на котором выполнялось тестирование:

\begin{itemize}[label=---]
	\item Операционная система: macOS Ventura 13.2.1 (22D68) \cite{macos};
	\item Память: 16 Гб с тактовой частотой 2133 МГц LPDDR3 \cite{memory};
	\item Процессор: Intel Core™ i7-8559U \cite{intel} с тактовой частотой  2.70 ГГц;
	\item Видеокарта: Intel Iris Plus Graphics 655 \cite{graphics} c объёмом памяти 1536 Мб.
\end{itemize}

Тестирование проводилось на ноутбуке, включенном в сеть электропитания. Во время тестирования ноутбук был нагружен только системой тестирования (работающим приложением) и системным окружением операционной системы.

\subsection{Сравнение времени обработки}

Результаты измерений  времени обработки запросов. выполненных при помощи чистого SQL, ORM и Query Builder, показаны на рисунке \ref{fig:research}.

\begin{figure}[h]
	\centering
	\captionsetup{justification=centering}
	\includegraphics[width=130mm]{img/research.png}
	\caption{Результаты замеров времени обработки запросов}
	\label{fig:research}
\end{figure}


Исходя из полученных результатов, выполнение запросов при помощи чистого SQL является наиболее быстрым способом (1.78 миллисекунды). Использование ORM является более эффективным, чем применение Query Builder (2.84 миллисекунды для ORM и 12.79 миллисекунды для Query Builder). При использовании ORM лишь 1.96 миллисекунды было затрачено на выполнение SQL, что лишь немного превышает время обработки запроса при помощи чистого SQL (разница составила 0.18 мс для ORM против 9.93 мс для Query Builder).

\subsection*{Вывод}
В данном разделе были описаны элементы интерфейса и приведены примеры работы разработанной программы. 
Графический интерфейс предоставляет возможность просмотреть список методов реализовнного API и их подробные характеристики (тип метода, адрес, используемые структуры), а также отправить конкретный запрос, осуществив ввод необходимых аргументов.
Проведён анализ времени обработки запросов с использованием чистого SQL, ORM и Query Builder, из результатов следует, что использование чистого SQL является самым быстрым способом (1.78 мс), однако не намного превосходит время, затраченное на выполнение SQL при использовании ORM (на 0.18 мс). Использование Query Builder занимает больше всего времени (12.79 мс).




